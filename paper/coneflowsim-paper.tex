\documentclass[11pt,a4paper]{article}

% ---------- Packages ----------
\usepackage[margin=2.5cm]{geometry}
\usepackage{amsmath,amssymb}
\usepackage{siunitx}
\usepackage{graphicx}
\usepackage{booktabs}
\usepackage[hidelinks]{hyperref}
\usepackage{bm}

\title{ConeFlowSim: an educational Python tool for supersonic flow over sharp cones}
\author{Seyed Arsham Asgari\thanks{Independent researcher. Email: s.a.asg2000@gmail.com}}
\date{\today}

\begin{document}
\maketitle

\begin{abstract}
ConeFlowSim is a small, open-source Python tool for studying supersonic flow over sharp cones.
The code couples two classical models: a weak oblique-shock (wedge) approximation and the
Taylor--Maccoll conical flow equations for a polytropic gas.
Despite its simplicity, ConeFlowSim produces physically meaningful Mach and pressure fields
between the cone surface and the conical shock, and provides side-by-side comparison plots
that are suitable for teaching and early-stage design studies.
This note summarizes the governing equations, numerical implementation, and example results,
and highlights the potential of ConeFlowSim as a reproducible reference implementation
for educational and research use.The source code of \texttt{ConeFlowSim} is openly available on Zenodo
(version~v1.0.1, DOI: \href{https://doi.org/10.5281/zenodo.17833232}{10.5281/zenodo.17833232}).
\end{abstract}

\section{Introduction}
Supersonic flow over sharp cones is a classical topic in compressible aerodynamics.
The flow field combines an axisymmetric conical shock with a nearly self-similar
subsonic or transonic region between the shock and the cone surface.
Analytical and semi-analytical models, such as the wedge (two-dimensional oblique-shock)
approximation and the Taylor--Maccoll conical flow equations, provide insight into the
structure of the flow and the sensitivity to Mach number, cone angle, and gas properties.

High-fidelity numerical simulations based on finite-volume or finite-element methods
are widely used for detailed design studies, but they are often too heavy for use in
introductory courses or rapid parameter studies. There is therefore value in
lightweight, script-level tools that implement simplified but physically meaningful
models and expose them through a transparent Python API.

This note presents \emph{ConeFlowSim}, an educational Python tool that:
(i) implements a weak oblique-shock approximation for a cone,
(ii) solves the Taylor--Maccoll equations using an explicit Runge--Kutta integrator, and
(iii) constructs two-dimensional Mach and pressure fields between the cone surface and
the conical shock, together with publication-quality plots for comparison between
the two models.

\subsection*{Use in teaching}

ConeFlowSim is primarily intended as an educational tool.  In a typical compressible-flow course it can be used to (i) visualise the structure of conical shock layers, (ii) explore the dependence of the flow field on Mach number and cone angle, and (iii) provide short computational projects where students modify or extend the Python code.  Because the code base is small and written in pure Python with only NumPy, SciPy and Matplotlib as dependencies, it is well suited for use in undergraduate laboratories and for self-study.


\section{Governing models}
We consider a perfect gas with ratio of specific heats $\gamma$ and an upstream
uniform supersonic flow with Mach number $M_1$.
The cone has half-angle $\theta_c$ and its apex coincides with the origin.
All lengths are scaled so that the axial extent of the computational domain is~$L$.

\subsection{Wedge / oblique-shock approximation}
For small cone angles it is common to approximate the axisymmetric conical flow by
a two-dimensional wedge-flow with deflection angle $\theta_c$.
Given $M_1$, $\theta_c$ and $\gamma$, the weak oblique-shock solution yields the
shock angle $\beta$ and the downstream Mach number $M_2$ together with the
pressure and temperature ratios across the shock.
Standard compressible-flow relations are used; ConeFlowSim collects these into a
function \texttt{oblique\_shock}.

To construct a simple field between the cone surface and the shock, the physical
domain is mapped to a normalized $(x,r)$ plane, where $x \in [0,1]$ is the
streamwise coordinate and $r$ is the radial coordinate normalized by $x$.
The cone surface corresponds to $r = \tan \theta_c$, and the shock to
$r = \tan \beta$.
The wedge model assumes that all flow quantities depend only on the polar angle
between these two boundaries, which leads to a piecewise-constant field of $M$
and $p/p_1$ between the cone and the shock.

\subsection{Taylor--Maccoll conical flow}
A more accurate description of the axisymmetric conical flow is obtained from
the Taylor--Maccoll equations, which follow from the steady Euler equations
under the assumption of self-similar conical flow.
Using a polar angle $\theta$ measured from the symmetry axis and a suitable
non-dimensionalization, the radial velocity component can be written in terms
of a dimensionless function $F(\theta)$.
For a polytropic gas, the governing second-order nonlinear ordinary differential
equation for $F$ can be written in the form
\begin{equation}
\bigl[ a(F;\gamma) \bigr] F''(\theta)
 = b(F,F';\theta,\gamma),
\label{eq:TM}
\end{equation}
where $a$ and $b$ collect the standard Taylor--Maccoll coefficients.
The local Mach number $M(\theta)$ and the static pressure $p(\theta)$ follow
from $F$ via the usual isentropic relations.

In practice, the ODE~\eqref{eq:TM} is rewritten as a first-order system for
the state vector $y=(F,F_\theta)$ and integrated from the shock angle
$\theta=\beta$ down to the cone angle $\theta=\theta_c$.
The initial conditions at the shock are obtained from the oblique-shock
relations using the assumption that the flow is aligned with the cone
surface at $\theta=\theta_c$.

\section{Numerical implementation}
ConeFlowSim is implemented in pure Python and organized as a small package
under \texttt{src/coneflowsim}. The numerical core relies only on
\texttt{NumPy}, \texttt{SciPy} and \texttt{Matplotlib}.

The Taylor--Maccoll ODE is integrated using an explicit classical
fourth-order Runge--Kutta scheme on a uniform grid in~$\theta$.
For typical parameters $(M_1,\theta_c)=(3.0,10^\circ)$ a few hundred steps
are sufficient to obtain smooth profiles of $M(\theta)$ and $p(\theta)/p_1$.
The wedge and Taylor--Maccoll solvers share a common upstream state and gas
model, which facilitates direct comparison between the two models.

The two-dimensional fields in the $(x,r)$ plane are constructed by discretizing
the streamwise coordinate $x \in [0,L]$ and interpolating the one-dimensional
profiles to the polar angle associated with each grid point.
The current implementation uses a simple nearest-neighbour mapping, which is
sufficient for educational plots and can easily be refined in future work.

\section{Example results}
All figures in this section were generated using the public \texttt{cone\_demo.py}
script distributed with ConeFlowSim. The example considers a cone with
half-angle $\theta_c = 10^\circ$ in a free stream with $M_1 = 3.0$ and
$\gamma = 1.4$.

\subsection{Taylor--Maccoll profiles}
Figure~\ref{fig:tm-profile} shows the static Mach number $M(\theta)$ obtained
from the Taylor--Maccoll solver as a function of the polar angle between the
shock and the cone surface. As expected, the Mach number decreases monotonically
from the post-shock value near $\theta=\beta$ to a subsonic value close to the
cone surface.

\begin{figure}[t]
  \centering
  \includegraphics[width=0.6\textwidth]{figures/tm_mach_profile.png}
  \caption{Taylor--Maccoll Mach number profile for $M_1=3.0$ and
  $\theta_c=10^\circ$.}
  \label{fig:tm-profile}
\end{figure}

\subsection{Wedge model fields}
Figure~\ref{fig:wedge-fields} presents the Mach number and pressure ratio
$p/p_1$ in the $(x,r)$ plane for the wedge approximation.
The cone surface and the shock are plotted as solid and dashed lines, respectively.
Although the fields are piecewise uniform, they already provide a useful visual
representation of the conical flow and the location of the shock.

\begin{figure}[t]
  \centering
  \includegraphics[width=\textwidth]{figures/wedge_fields.png}
  \caption{Mach number and pressure ratio $p/p_1$ between the cone surface and
  the shock for the wedge model.}
  \label{fig:wedge-fields}
\end{figure}

\subsection{Comparison between wedge and Taylor--Maccoll fields}
Finally, Fig.~\ref{fig:comparison} compares the wedge and Taylor--Maccoll
fields side by side for both Mach number and pressure ratio.
The wedge model tends to overpredict the Mach number in the interior of the
flow region and underpredict the pressure on the cone surface, especially for
larger cone angles. The Taylor--Maccoll solution captures the smooth variation
of the flow quantities and can therefore serve as a more accurate reference
for educational and design purposes.

\begin{figure}[t]
  \centering
  \includegraphics[width=\textwidth]{figures/wedge_vs_tm_fields.png}
  \caption{Side-by-side comparison of Mach number and pressure ratio $p/p_1$
  for the wedge and Taylor--Maccoll models.}
  \label{fig:comparison}
\end{figure}

\section{Reuse potential and educational value}
ConeFlowSim is intentionally small and readable; the entire Taylor--Maccoll
solver fits in a few dozen lines of Python.
This makes the code suitable for use in graduate-level courses on compressible
flow, as well as for self-study and for rapid feasibility studies in early
design phases.

Possible extensions include:
\begin{itemize}
  \item incorporating real-gas effects or temperature-dependent $\gamma$;
  \item adding graphical user interfaces or Jupyter widgets for interactive use;
  \item coupling the conical flow model to simple aerodynamic force estimates.
\end{itemize}

\section{Conclusions}
This note has presented ConeFlowSim, an open-source Python tool for studying
supersonic flow over sharp cones using both a wedge approximation and the
Taylor--Maccoll conical flow equations.
The tool provides transparent reference implementations of classical models
and generates publication-quality plots that can be used directly in teaching
material and research articles.
Future work will focus on extending the range of models and adding more
benchmark cases.

\section*{Acknowledgements}
The author gratefully acknowledges the use of open-source scientific Python
software and the feedback from early users of ConeFlowSim.

\section{Summary and outlook}

We have presented ConeFlowSim, a small open-source Python package for studying supersonic flow over sharp cones using a weak oblique-shock approximation and the classical Taylor--Maccoll equations.  Despite its simplicity, the code produces physically meaningful Mach and pressure fields between the cone surface and the conical shock and provides side-by-side comparison plots that are well suited for teaching compressible-flow concepts.  Possible extensions include adding more advanced conical-flow models, coupling to boundary-layer solvers, or interfacing ConeFlowSim with high-fidelity CFD data for validation exercises.


\begin{thebibliography}{9}
	
	\bibitem{anderson}
	J.~D. Anderson,
	\emph{Modern Compressible Flow: With Historical Perspective}
	(McGraw--Hill, 3rd ed., 2003).
	
	\bibitem{taylor_maccoll}
	G.~I. Taylor and J.~W. Maccoll,
	``The solution of the flow past a cone using the method of characteristics,''
	\emph{Proc. R. Soc. Lond. A} \textbf{139}, 278--311 (1933).
	
	\bibitem{liepmann}
	H.~W. Liepmann and A. Roshko,
	\emph{Elements of Gasdynamics}
	(Dover, 2001).


\end{thebibliography}


\end{document}
